\section{Discussion}

Our theoretical and empirical results provide several key insights into the behavior of momentum-based stochastic gradient descent under heavy-tailed noise conditions. The main theoretical result establishes a linear convergence rate of $(1 - \mu\eta(1-\beta))^k$ with an additive noise term $O(\eta\sigma^2/\mu)$, which is particularly significant as it demonstrates robustness to non-Gaussian noise distributions.

The experimental findings strongly support the theoretical guarantees. The observed linear convergence under both Gaussian and $t$-distributed noise validates our theoretical framework, while the minimal difference in performance between these noise types suggests that the method's effectiveness is not significantly impacted by heavy-tailed perturbations. This robustness is particularly valuable for practical applications where noise distributions may deviate from Gaussian assumptions.

The role of the momentum parameter $\beta$ emerges as a crucial factor in the convergence behavior. Higher values of $\beta$ lead to more stable trajectories at the cost of slower convergence, consistent with the $(1-\beta)$ term in our theoretical rate. This trade-off provides practitioners with a tunable parameter to balance convergence speed against stability requirements.

The asymptotic behavior warrants special attention. The convergence to a noise ball of radius proportional to $O(\eta\sigma^2/\mu)$ aligns precisely with our theoretical predictions. This phenomenon, observed consistently across different noise distributions, suggests that the limiting factor in optimization accuracy is indeed the noise variance $\sigma^2$ rather than the specific shape of the noise distribution.

These findings have several important implications:

1. The method's robustness to heavy-tailed noise suggests it can be reliably applied in settings where Gaussian noise assumptions fail.

2. The explicit dependence on the momentum parameter $\beta$ provides clear guidance for parameter tuning in practical applications.

3. The tight correspondence between theoretical and empirical results validates our analysis framework and suggests the bounds are nearly sharp.

One limitation of our analysis is that we assume the Polyak-Łojasiewicz condition, which may not hold in all practical scenarios. Future work could explore relaxing this assumption while maintaining similar convergence guarantees. Additionally, investigating adaptive momentum schemes that automatically adjust $\beta$ based on the observed noise characteristics could further improve practical performance.