\section{Limitations and Future Work}

While our analysis provides theoretical guarantees for gradient descent convergence rates, several important limitations should be acknowledged:

\subsection{Theoretical Assumptions}

The primary limitation stems from the relatively strong assumptions required for our theoretical analysis. Most notably:

\begin{itemize}
    \item The Polyak-Łojasiewicz (PL) inequality assumption ($\exists \mu > 0$ such that $\frac{1}{2}\|\nabla f(x)\|^2 \geq \mu(f(x) - f^*)$) is crucial for our convergence results but fails to hold for many practical optimization problems, particularly in non-convex settings common in deep learning.
    
    \item Our requirement of $L$-Lipschitz continuous gradients ($\|\nabla f(x) - \nabla f(y)\| \leq L\|x-y\|$) excludes important non-smooth optimization problems that frequently arise in practice, such as those involving $\ell_1$ regularization or ReLU activation functions.
    
    \item The constant variance assumption $\mathbb{E}\|\nabla f(x) - \nabla F(x)\|^2 \leq \sigma^2$ may be overly simplistic, as noise levels typically vary across different regions of the optimization landscape.
\end{itemize}

\subsection{Practical Considerations}

The theoretical convergence rates we derive ($\mathcal{O}(1/k)$ for strongly convex functions and $\mathcal{O}(1/\sqrt{k})$ for convex functions) may not reflect actual performance in practical applications where:

\begin{itemize}
    \item The choice of step size $\eta_k$ requires knowledge of problem-specific parameters ($L$ and $\mu$) that are often unknown or computationally expensive to estimate in practice.
    
    \item Our analysis assumes infinite precision arithmetic, whereas practical implementations face numerical precision limitations that may affect convergence behavior.
\end{itemize}

\subsection{Future Research Directions}

Several promising directions for future work include:

\begin{enumerate}
    \item Extending the analysis to weaker geometric conditions than the PL inequality, such as the Kurdyka-Łojasiewicz inequality or other relaxed smoothness conditions.
    
    \item Developing adaptive step size schemes that maintain theoretical guarantees without requiring knowledge of problem parameters.
    
    \item Analyzing convergence behavior under more realistic noise models where $\sigma^2$ varies with the location in parameter space.
    
    \item Investigating the impact of finite precision arithmetic on theoretical convergence guarantees.
\end{enumerate}

These limitations suggest that while our results provide valuable theoretical insights, bridging the gap between theory and practice remains an important challenge in optimization research.